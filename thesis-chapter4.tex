\chapter{Experiments}

\section{Toy Model Simulator}
Our toy model is a very basic simulator that traces the paths of virtual gamma-ray photons through our detector. This allows us to test code in a controlled environment, which is very useful in the development of our reconstruction algorithm and enables us to ensure our code is performing as we expect without the additional errors caused by outside factors. However, due to this simplicity, there will be physical effects present in the real world that we fail to account for in our simulator, such as background radiation, photon scatters in the detector support material, and various electronic effects, that will increase our overall error. We do not take the detector material into account, nor do we allow for any types of interactions other than a series of Compton scatters followed by a photoabsorption. We set up a test framework on this model simulator that will allow us to adapt our testing to more complicated gamma-ray simulators in the future.

\subsection{Photon simulation} \label{toy_model}
The simulator function, with pseudo-code shown in Algorithm \ref{alg:recon}, takes as its parameters the number of photons to generate and the number of hits each photon should have (including the final absorption). We pick the initial energy of each photon from a uniform distribution between 100 keV and 20 MeV (Line \ref{ln:initE}).

\begin{algorithm}
\caption{Toy Model Simulator}\label{recon}
\hspace{\algorithmicindent} \textbf{Constants:} \texttt{minE}, \texttt{maxE} = 100 keV, 20 MeV - limits on initial photon energy
\hspace{\algorithmicindent} \textbf{Inputs:} \texttt{numEvents} - the number of photons to simulate, \texttt{numHits} - hits per event, \texttt{events} and \texttt{results} - empty arrays for storage
\begin{algorithmic}[1]
\Function{simulate}{numEvents, numHits, events[], results[]}
    \State init alphas[], W[], k[], x[] \Comment{Arrays for the angles, energies, directions,}
    \State \Comment{and hit positions, respectively}
    \For{i in numEvents}
        \State // Initialize variables:
        \State Result \& r $\leftarrow$ results[i] \Comment{Set up result and event references}
        \State Event \& e $\leftarrow$ events[i]
        \State e.hits $\leftarrow$ empty \texttt{Hit} array \Comment{Initialize \texttt{Event} hits}
        \State e.numHits $\leftarrow$ numHits
        \State
        \State // Generate initial energy at random:
        \State r.energy $\leftarrow$ uniform\_random(minE, maxE) \Comment{E $\in$ [minE maxE]} \label{ln:initE}
        \State
        \State // Generate n-1 random angles:
        \State alphas $\leftarrow$ uniform\_random($-\frac{\pi}{2}$, $\frac{\pi}{2}$, n-1) \Comment{Angle array of length n-1} \label{ln:alphas}
        \State r.eta $\leftarrow$ cos(alphas[0]) \Comment{Save initial angle}
        \State 
        \State // Calculate \texttt{Hit} positions, directions, and energies from angles (see \S\ref{toy_model})
        \State calcW(numHits, r.energy, alphas[], W[]) \Comment{Stores energy deposits in W[]} \label{ln:calcW}
        \State calcK(numHits, alphas[], k[]) \Comment{Stores photon directions in k[]} \label{ln:calcK}
        \State calcX(numHits, k[], x[]) \Comment{Stores \texttt{Hit} positions in x[]} \label{ln:calcX}
        \State
        \State perm $\leftarrow$ random permutation of x[] indices \Comment{Randomly shuffle hits} \label{ln:shuffle}
        \State r.p0, r.p1 $\leftarrow$ perm[0], perm[1] \Comment{First two hit indices}
        \State 
        \State // Record each \texttt{Hit}'s position and energy deposit
        \For{j in numHits}
            \State E\_dep $\leftarrow$ (W[j] - W[j+1])*$mc^2$ \Comment{Calculate energy deposit}
            \State e.hits[perm[j]] $\leftarrow$ (x[j], E\_dep) \Comment{Save \texttt{Hit}}
        \EndFor
        \State
        \State addNoise(e) \Comment{Add measurement noise to each hit in the \texttt{Event}} \label{ln:noise}
        \State
    \EndFor
\EndFunction
\end{algorithmic}
\end{algorithm}

Once we have initialized our photon, we simulate its interaction with the detector by generating a random set of scattering angles according to the input number of hits (Line \ref{ln:alphas}). The scattering angles are taken from a uniform distribution from -90 to +90 degrees. We do not simulate back-scatters ($\geq$ 90 degrees), since these are relatively unlikely in the real world. We then use formula \ref{eq:compton} to calculate the energy deposited for each generated scattering angle and store these values in an array (Line \ref{ln:calcW}, \texttt{calcW}).

After calculating the angles and energy deposits, we then must calculate the position of each hit in the detector. To do this, we need to first determine which layers of the detector the photons will scatter in and the vector directions between each pair of hits. This takes place in our function \texttt{calcK} on line \ref{ln:calcK} of our algorithm: for each scatter, we determine the layer of each hit based on an exponential probability distribution, calculate the scattering direction vector by adding the scattering angle (with a random $\phi$ angle) to the photon's previous direction, and save the direction vectors in array \texttt{k}. We can then use simple vector addition to determine the \texttt{Hit} positions and save them in array \texttt{x} (Line \ref{ln:calcX}, \texttt{calcX}). We randomly shuffle the hits (Line \ref{ln:shuffle}) in order to simulate the unknown ordering of hits in the detector. If we saved the hits in sequential order for each photon it might cause our program runtime to be shorter than is realistic. Our last step, on Line \ref{ln:noise} of our pseudocode, is to simulate some random detector noise and add it to each position and energy measurement. The distributions and parameters we use to generate this noise are discussed in the next section.

\subsection{Errors and noise} \label{noise}
We simulate noise in our detector using a Gaussian distribution with variable standard deviation, $\sigma_S$. For the position measurements, the user inputs the standard deviation in millimeters, and for the energy measurements, the standard deviation is input as a constant factor of the energy, $a$:
\begin{equation}\label{eq:sigE}
\frac{\sigma_E}{E} = a*\frac{1}{E \; (keV)}
\end{equation}
We expect based on previous tests of the detector material that $\sigma_S$ will be about 1 mm and $a$ will be about 0.22.

Though the uniform distribution we generate our scattering angles from does not reflect the correct Compton cross-section, our use of Formula \ref{eq:compton} ensures that no interaction we generate would be impossible in nature. As our reconstruction algorithm only compares the inferred angles to the true angles and does not take the scattering cross-section into account either, this simulator still provides a fairly good test of our performance in the absence of systematic measurement error. The true distribution of scattering angles is determined by the Klein-Nishina formula and the interaction probability by the Thomson cross-section \cite{klein-nishina}, so in later iterations of this project it may be useful to incorporate these into our toy model, particularly if we choose to integrate some component of the true scattering probabilities in our reconstruction algorithm.

\section{Hardware}
The hardware we use to test our algorithm is the Raspberry Pi 3, Model B+. It uses a 4-core ARM processor with a 1.4 GHz processing speed, and has 1 GB of RAM. Due to its low number of cores and static execution, it is a slower processor than found in most laptops, but it is perfect for testing our software as it is low-power enough to be used as part of a space telescope. The ARM processor is commonly used for scientific applications, and because NASA has recently commissioned a space processor based on the same architecture, the Raspberry Pi could be the closest hardware available to what our program will actually be running on in space. We tested our parallelism on Cassini, a more advanced machine at Washington University, to get a more accurate measure of our core speedup in Section \ref{parallel}.

\section{Performance Measures}
We use two main performance measures to evaluate our algorithm - the number of photons processed per second, and the accuracy. We consider a reconstruction accurate if the first two hits of the reconstruction match the first two hits in the correct sequence, and refer to this measure as the "two-hit accuracy", or simply the accuracy. We report this statistic as the number of accurate reconstructions over the number of total reconstructions. We also consider power consumption to be a measure of performance, but any trends in power are not studied in-depth as we are well below our expected limit.

\section{Parameters tested} \label{vars}
There are many variables that have the potential to affect the algorithm's performance, and we want to specifically examine the trade-offs in reconstruction speed vs. accuracy for several key parameters. The tests we run are as follows:

\begin{enumerate}
    \item Power - We use the WITRN U2 USB Power Monitor to test the power used by the Raspberry Pi both when it is idle and when it is running our program.
    \item Single vs. double precision - As our program is written in C++, we have a choice of whether to use \texttt{float} values (single precision) or \texttt{double} (double-precision) values in our calculations. Single precision values use fewer bytes than double values, making them faster to use in calculations but sometimes less accurate due to rounding error.
    \item Simulated hits - The number of hits we simulate for each event.
    \item Reconstructed hits - The number of hits we use to reconstruct a photon's trajectory. We cut off computation once the specified number of hits is reached, after which the sequence with the lowest $\chi^2$ value so far is presumed to be the correct one. (i.e. we set a lower-depth base case in Line \ref{ln:hitsUsed} of Algorithm \ref{alg:recurse}).
    \item $\chi^2$ cutoff - The p-value we use for our cutoff $\chi^2$ values. Ex: a p-value of .1 would give us a 10\% probability of cutting off each good triple, whereas a p-value of .01 would give us a 1\% probability of doing so.
    \item $\eta'$-tolerance - The amount above 1 or below -1 that we will allow for a calculated cos(energy angle). For example, if the $\eta'$-tolerance is 0.2, any sequences containing an $\eta' > 1.2$ or $< -1.2$ would be cut.
    \item Predicted spatial and energy noise levels - Our $\chi^2$ calculations rely on the predicted errors of our $\eta$ and $\eta$ values, so changing the predicted level of error will change the behavior of our algorithm. We vary the $\sigma_E$ and $a$ values found in Section \ref{noise} and investigate the results.
    \item Simulated spatial and energy noise - We investigate the interplay between the predicted noise and the simulated noise in both the spatial and energy regimes. Though we have no control over the noise levels we will encounter in operation, it is useful to simulate a range of noise levels and adjust our predicted noise parameters accordingly. This is also useful as a proof-of-concept, as we can check to make sure our algorithm is behaving as expected under increased noise thresholds.
\end{enumerate}