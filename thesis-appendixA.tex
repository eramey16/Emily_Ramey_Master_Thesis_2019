\chapter{Pseudocode}
\label{app:code}

%%% Sequential & iterative algorithm
\begin{algorithm}
\caption{Sequential Reconstruction Algorithm Part 1}\label{recon1}
\begin{algorithmic}
\For {each photon}
    \State $n \gets$ \# hits for photon $i$
    \State min$\chi^2 \gets$ max double
    \State bestPermutation $\gets 0$
    \For {permutation $j \in$ all $n!$ permutations}
        \State $\chi^2 \gets 0$
        \For {hit $k\in 1...n-1$}
            \State // Calculate spatial angle:
            \State $\vec{x}_k \gets$ position of hit k
            \State $\hat{r}_k \gets |\vec{x}_{k+1} - \vec{x}_k|$ \Comment{Unit vector along photon direction}
            \State $\eta_k = \cos{\phi_k} = \hat{r}_k\cdot\hat{r}_{k-1}$ \Comment{From dot product equation}
            \State
            \State // Calculate energy angle:
            \State $W_k = \frac{1}{m_e c^2}\sum_{i=k+1}^n E_i$ \Comment{Unitless energy; energy before hit k}
            \State $\eta'_k = \cos{\phi'_k} = 1 + \frac{1}{W_{k-1}}-\frac{1}{W_k}$ \Comment{From Compton equation}
            \State
            \State // Calculate Errors:
            \State $\delta\eta_k^2 = \delta\phi_{k,r}^2\sin^2(\phi_k)$ \Comment{$\delta\phi_{k,r}$ depends on $\delta x,\delta y,\delta z$}
            \State $\delta\eta'^2_k = \frac{\delta W_{k-1}^2}{W_{k-1}^4}+\delta W_k^2 \big[(\frac{1}{W_k^2}-\frac{1}{W_{k-1}^2})^2 - \frac{1}{W_{k-1}^4}\big]$
            \State
            \State // Add to $\chi^2$:
            \State $\chi^2 \gets \chi^2 + \frac{1}{n-2} \frac{(\eta_k-\eta'_k)^2}{\delta\eta_k^2 + \delta\eta'^2_k}$
        \EndFor
        \algstore{break}
\end{algorithmic}
\end{algorithm}
    
\begin{algorithm}
\caption{Sequential Reconstruction Algorithm Part 2}\label{recon2}
\begin{algorithmic}
        \algrestore{break}
        \State
        \If{$\chi^2 <$ min$\chi^2$}
        \State min$\chi^2 \gets \chi^2$
        \State bestPermutation $\gets j$
        \EndIf
    \EndFor
    \State
    \State // Predict $\eta$:
    \State $\eta_{pred} = 1 + \frac{1}{W_0}-\frac{1}{W_1} \pm \delta\eta'_1$
    \State
\EndFor
\end{algorithmic}
\end{algorithm}

\iffalse

While this guide answers most questions about how to format a thesis, it does
not address questions about English grammar, use of abbreviations, punctuation,
spelling, and other confusing subjects.  Students should obtain a dictionary
and a style of grammar book to refer to as questions arise.  The dictionary is
important because most electronic spelling checkers are not complete and do not
contain definitions.  (You may also need to refer to some of the references you
cite for the spelling of technical terms.)  The grammar or style book is useful
for checking grammar and punctuation rules.  A good style manual contains
information about correct English usage as well as advice for preparing a
manuscript.  \textit{A Manual for Writers of Term Paper, Theses, and
Dissertations}~\cite{Turabian} is one such concise and inexpensive manual based
on the lengthy and more expensive \textit{Chicago Manual of
Style}~\cite{ChicagoManual}.  

The following rules will help you avoid three mistakes frequently made
by students:
\begin{itemize} 
 \item Hyphenated words must begin and end on the same page.

 \item When a page break falls in the middle of a paragraph, at least
two lines of text from that paragraph must appear on the second page.

 \item At least one line of text from a section or subsection must
appear on the same page as the title of that section or subsection.
\end{itemize}

%%% Local Variables: 
%%% mode: latex
%%% TeX-master: "thesis-main"
%%% End: 
\fi